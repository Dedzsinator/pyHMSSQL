\documentclass{beamer}
\usepackage[utf8]{inputenc}
\usepackage[hungarian]{babel}
\usepackage{graphicx}
\usepackage{listings}
\usepackage{xcolor}

\usetheme{Madrid}
\usecolortheme{default}

\title{HMSSQL - Adatbázis-kezelő Rendszer}
\subtitle{SQL Feldolgozási Pipeline és Belső Működés}
\author{Brustur-Buksa Beatrice, Dégi Nándor, Erdei Kristóf, Komjátszegi-Fábián Hunor}
\date{\today}

\lstset{
    basicstyle=\ttfamily\footnotesize,
    backgroundcolor=\color{gray!10},
    frame=single,
    breaklines=true
}

\begin{document}

\frame{\titlepage}

% Slide 1
\begin{frame}
\frametitle{Bevezetés - HMSSQL ABKR Rendszer}

\textbf{A rendszerről:}
\begin{itemize}
    \item Teljes SQL feldolgozási pipeline
    \item B+ fa indexekkel optimalizált lekérdezések
\end{itemize}

\vspace{0.5cm}
\textbf{Főbb komponensek:}
\begin{itemize}
    \item \textbf{Parser} - SQL szintaxis elemzés
    \item \textbf{Execution Engine} - Lekérdezések végrehajtása
    \item \textbf{Buffer Pool Manager} - Memória és lemez közötti adatkezelés
    \item \textbf{Index Manager} - B+ fa indexek kezelése és optimalizálása
    \item \textbf{Optimizer} - Lekérdezési tervek költség-alapú optimalizálása
\end{itemize}

\end{frame}

% Slide 2
\begin{frame}
\frametitle{Architektúra Áttekintés}

\textbf{Rendszer felépítése:}
\begin{center}
\texttt{SQL Query → Parser → Planner → Optimizer → Execution Engine → Storage}
\end{center}

\vspace{0.5cm}
\textbf{Kulcs komponensek:}
\begin{itemize}
    \item \textbf{haskell\_parser.py} - Jelenleg Python-ban
    \item \textbf{planner.py} - Lekérdezési terv készítés
    \item \textbf{optimizer.py} - Költség-alapú optimalizálás
    \item \textbf{execution\_engine.py} - Végrehajtási motor
    \item \textbf{buffer\_pool\_manager.py} - Memória kezelés
    \item \textbf{index\_manager.py} - Index kezelés
\end{itemize}

\end{frame}

% Slide 3
\begin{frame}
\frametitle{SQL Parancs Végrehajtási Pipeline}

\textbf{1. Parsing Fázis:}
\begin{itemize}
    \item Haskell parser elemzi a SQL-t
    \item AST (Abstract Syntax Tree) létrehozása
    \item Szintaxis ellenőrzés
\end{itemize}

\vspace{0.3cm}
\textbf{2. Planning Fázis:}
\begin{itemize}
    \item Logikai terv készítése
    \item Tábla és oszlop validálás
    \item Join sorrend meghatározása
\end{itemize}

\vspace{0.3cm}
\textbf{3. Optimization Fázis:}
\begin{itemize}
    \item Join algoritmusok kiválasztása
    \item Index kiválasztás
    \item Predikátum push-down
\end{itemize}

\end{frame}

% Slide 4
\begin{frame}
\frametitle{Execution Engine Működése}

\textbf{Végrehajtási lépések:}
\begin{enumerate}
    \item \textbf{Operator inicializálás} - Scan, Join, Aggregation operátorok
    \item \textbf{Iterator modell} - Volcano-style végrehajtás
    \item \textbf{Tuple-by-tuple feldolgozás} - Memória hatékony
    \item \textbf{Pipeline execution} - Streaming feldolgozás
\end{enumerate}

\vspace{0.5cm}
\textbf{Operator típusok:}
\begin{itemize}
    \item \texttt{TableScan} - Tábla beolvasás
    \item \texttt{IndexScan} - Index alapú keresés
    \item \texttt{NestedLoopJoin} - Beágyazott ciklusú join
    \item \texttt{HashJoin} - Hash alapú join
    \item \texttt{Aggregation} - Csoportosítás és összesítés
\end{itemize}

\end{frame}

% Slide 5
\begin{frame}
\frametitle{Buffer Pool Manager}

\textbf{Memória kezelés stratégiája:}
\begin{itemize}
    \item \textbf{LRU (Least Recently Used)} - Legrégebben használt kiszorítása
    \item \textbf{Pin/Unpin mechanizmus} - Aktív oldalak védése
    \item \textbf{Dirty page tracking} - Módosított oldalak nyomon követése
    \item \textbf{Background flushing} - Aszinkron lemezre írás
\end{itemize}

\vspace{0.5cm}
\textbf{Optimalizációk:}
\begin{itemize}
    \item Előre betöltés (prefetching)
    \item Batch I/O műveletek
    \item Memória pool újrafelhasználás
\end{itemize}

\end{frame}

% Slide 6
\begin{frame}
\frametitle{B+ Fa Index Tárolás}

\textbf{bptree.py és bptree\_optimized.pyx implementáció:}
\begin{itemize}
    \item \textbf{Leaf node-ok} - Tényleges adatok tárolása
    \item \textbf{Internal node-ok} - Routing információ
    \item \textbf{Cython optimalizáció} - C szintű teljesítmény
\end{itemize}

\vspace{0.3cm}
\textbf{Tárolási stratégia:}
\begin{itemize}
    \item \textbf{Page-based storage} - Fix méretű blokkok
    \item \textbf{Split/Merge algoritmusok} - Automatikus kiegyensúlyozás
    \item \textbf{Bulk loading} - Hatékony kezdeti betöltés
    \item \textbf{Range queries} - Intervallum lekérdezések optimalizálása
\end{itemize}

\vspace{0.3cm}
\textbf{Különlegességek:}
\begin{itemize}
    \item bptree\_visualizer.py - Debug és monitoring
    \item Adaptív node méret
\end{itemize}

\end{frame}

% Slide 7
\begin{frame}
\frametitle{Aggregációs Függvények Implementálása}

\textbf{Aggregation Engine:}
\begin{itemize}
    \item \textbf{Hash-based grouping} - GROUP BY hatékony kezelése
    \item \textbf{Streaming aggregation} - Memória-hatékony feldolgozás
    \item \textbf{Parallel aggregation} - Többszálú összeG számítás
\end{itemize}

\vspace{0.3cm}
\textbf{Támogatott függvények:}
\begin{itemize}
    \item \texttt{COUNT, SUM, AVG} - Alapvető statisztikák
    \item \texttt{MIN, MAX} - Szélsőértékek
    \item \texttt{DISTINCT} - Egyedi értékek számlálása
\end{itemize}

\vspace{0.3cm}
\textbf{Optimalizációk:}
\begin{itemize}
    \item Pre-aggregation - Korai csoportosítás
    \item Index-based MIN/MAX - Index alapú szélsőérték keresés
    \item Incremental aggregation - Növekményes számítás
\end{itemize}

\end{frame}

% Slide 8
\begin{frame}
\frametitle{10K Rekord SELECT Kezelése}

\textbf{Nagy lekérdezések optimalizálása:}
\begin{itemize}
    \item \textbf{Index scan} - B+ fa alapú hatékony keresés
    \item \textbf{Batch processing} - Blokk szintű feldolgozás
    \item \textbf{Pipeline execution} - Streaming feldolgozás
    \item \textbf{Buffer pool} - Intelligens memória kezelés
\end{itemize}

\vspace{0.3cm}
\textbf{Teljesítmény stratégiák:}
\begin{itemize}
    \item \textbf{Predicate pushdown} - Szűrés korai alkalmazása
    \item \textbf{Projection pushdown} - Csak szükséges oszlopok
    \item \textbf{Parallel execution} - Többszálú végrehajtás
    \item \textbf{Result caching} - Eredmény gyorsítótárazás
\end{itemize}

\vspace{0.3cm}
\textbf{Benchmark eredmények:}
\begin{itemize}
    \item benchmark.py - Teljesítmény mérés
    \item Automatikus skálázás (scaler.py)
\end{itemize}

\end{frame}

% Slide 9
\begin{frame}
\frametitle{Speciális Funkciók és Érdekességek}

\textbf{Optimizer finomságok:}
\begin{itemize}
    \item \textbf{Cost-based optimization} - Statisztika alapú döntések
    \item \textbf{Join order optimization} - Optimal join sorrend
    \item \textbf{Index selection} - Automatikus index kiválasztás
\end{itemize}

\vspace{0.3cm}
\textbf{Monitoring és Debug:}
\begin{itemize}
    \item \textbf{profiler.py} - Teljesítmény profilozás
    \item \textbf{table\_stats.py} - Tábla statisztikák
    \item \textbf{bptree\_visualizer.py} - Index struktúra vizualizáció
\end{itemize}

\vspace{0.3cm}
\textbf{REST API integrálás:}
\begin{itemize}
    \item \textbf{rest\_api.py} - HTTP-based interface
    \item \textbf{server.py} - Aszinkron szerver
    \item JSON-based kommunikáció
\end{itemize}

\end{frame}

% Slide 10
\begin{frame}
\frametitle{Összefoglalás és Jövőbeli Tervek}

\textbf{Rendszer erősségei:}
\begin{itemize}
    \item \textbf{Moduláris architektúra} - Könnyen bővíthető
    \item \textbf{Hatékony indexelés} - B+ fa optimalizációk
    \item \textbf{Intelligens optimalizáló} - Költség-alapú döntések
    \item \textbf{Skálázható design} - Nagy adatmennyiség kezelése
\end{itemize}

\vspace{0.3cm}
\textbf{Teljesítmény jellemzők:}
\begin{itemize}
    \item 10K+ rekord hatékony kezelése
    \item Streaming aggregation
    \item Memória-optimalizált buffer pool
    \item Cython-accelerated index operations
\end{itemize}

\vspace{0.3cm}
\textbf{Fejlesztési irányok:}
\begin{itemize}
    \item Parallel query execution
    \item Distributed query processing
    \item Advanced caching strategies
\end{itemize}

\end{frame}

\end{document}